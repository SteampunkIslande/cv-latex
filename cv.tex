\documentclass[10pt,a4paper,roman]{moderncv}
\moderncvtheme[blue]{classic}

\photo[80pt][0pt]{Charles.jpg}

\usepackage[utf8]{inputenc}
\usepackage{fontawesome}
\usepackage[scale=0.9]{geometry} % La taille pris par le contenu, ici on a 10% de marges.
\nopagenumbers{} % Permet de masquer les numéros de page

\setlength{\hintscolumnwidth}{0.25\textwidth}

\title{Ingénieur bioinformatique}
\firstname{Charles}
\familyname{MONOD-BROCA}
\address{39, avenue de Breteuil}{75007 Paris}{France}
\mobile{+33 6 35 92 44 97}
\email{charles.monod-broca@neuf.fr}
 

\begin{document}
\maketitle

\section{Expériences professionnelles}

\cventry{Mars 2021 - Juin 2021}{Stagiaire en bioinformatique}{Université de Bretagne Occidentale}{Brest}{}{Aide au développement d'un logiciel d'exploration des variants. Le logiciel, cutevariant, a fait l'objet d'un papier publié dans Bioinformatic advances.}

\cventry{Mars 2020 - Juin 2020}{Stagiaire en biochimie}{Muséum National d'Histoire Naturelle}{Paris}{}{Stage bibliographique (pandémie oblige) sur l'étude des peptides antimicrobiens et leurs effets sur le microbiote intestinal.}

\cventry{Janvier 2019 - Juin 2019}{Stagiaire en génétique moléculaire}{Institut Pasteur}{Paris}{}{Extraction d'ADN de S. pombe et détection de la méthylation en sortie de G0. Techniques utilisée: extraction d'ADN au phénol, FACS, Migration d'ADN sur gradient de sucrose}


\section{Formation}

\cventry{Septembre 2020 - Juin 2021}{Master - Digital Sciences}{Centre de Recherche Interdisciplinaire - Université de Paris}{Paris}{}{Machine learning, Intelligence Artificielle, Réalité augmentée/virtuelle, Robotique}

\cventry{Septembre 2019 - Juin 2020}{Master - Molécules et Cibles Thérapeutiques}{Muséum National d'Histoire Naturelle - Sorbonne Université}{Paris}{}{Biochimie, pharmacologie}

\cventry{Septembre 2017 - Juin 2019}{Master - Génétique et Epigénétique - Bases moléculaires de l'oncogénèse}{Institut Pasteur - Sorbonne Université}{Paris}{}{Cancérologie, génétique, épigénétique, oncogénèse}

\cventry{Septembre 2015 - Juin 2017}{Licence de Sciences de la Vie}{Université de Tours}{Tours}{}{Bases en biologie moléculaire}


\section{Compétences informatiques}
\cvcomputer{Langages}{Python\\C/C++\\C\#\\Jinja2\\\LaTeX\\}{Logiciels}{Git\newline{}Make\newline{}Snakemake\newline{}Jupyter\newline{}Blender\newline{}Unity\newline{}Suites bureautiques\newline{}}
\cvcomputer{Technologies}{Qt\newline{}Pytorch\newline{}Sqlite\newline{}Jinja2\newline{}}{}{}

%
%\cvline{Français}{Natif}
%
%\cvline{Anglais}{C1}
%
%\cvline{Espagnol}{B2}
%
\section{Centres d'intérêt}

\cvline{Loisirs}{Guitare\newline{}Astrophotographie\newline{}Arduino\newline{}Minecraft\newline{}Modélisation 3D\newline{}}

\end{document}