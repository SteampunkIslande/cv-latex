\documentclass[10pt,a4paper,roman]{moderncv}
\moderncvtheme[blue]{classic}
\usepackage[utf8]{inputenc}
\usepackage[scale=0.9]{geometry} % La taille pris par le contenu, ici on a 15% de marges.
\nopagenumbers{} % Permet de masquer les numéros de page

\setlength{\hintscolumnwidth}{0.175\textwidth}

\title{Ingénieur bioinformatique}
\firstname{Charles}
\familyname{MONOD-BROCA}
\address{39, avenue de Breteuil}{75007 Paris}
\mobile{+33 6 35 92 44 97}
\email{charles.monod-broca@neuf.fr}
 

\begin{document}
\maketitle

\section{Formation}

\cventry{Sep 2020 - Jun 2021}{Master - Digital Sciences}{Centre de Recherche Interdisciplinaire - Université de Paris}{Paris}{}{Ce master m'a donné les bases des techniques d'apprentissage automatique, à travers scikitlearn et pytorch. D'autres modules m'ont fait découvrir la robotique et la réalité virtuelle.}

\cventry{Sep 2019 - Jun 2020}{Master - Molécules et Cibles Thérapeutiques}{Muséum National d'Histoire Naturelle - Sorbonne Université}{Paris}{}{Ce master m'a donné des bases théoriques solides en pharmacologie (biochimie et biologie moléculaire). Les sujets abordés en cours allaient des maladies infectieuses au cancer.}

\cventry{Sep 2017 - Jun 2019}{Master - Génétique et Epigénétique - Bases moléculaires de l'oncogénèse}{Institut Pasteur - Sorbonne Université}{Paris}{}{Ce master orienté recherche m'a donné de bonnes bases théoriques dans la compréhension des mécanismes génétiques qui mènent au cancer.}

\cventry{Sep 2015 - Jun 2017}{Licence de Sciences de la Vie}{Université de Tours}{Tours}{}{Cette licence m'a permis de valider les prérequis en biologie moléculaire et cellulaire en vue de l'obtention d'un diplôme de Master.}

\cventry{Sep 2013 - Jun 2015}{CPGE BCPST}{Lycée Pothier}{Orléans}{}{Ces deux années de classe préparatoire dite agro/véto m'ont formé à l'apprentissage de nouvelles connaissances, à la curiosité et à la méthode scientifique.}


\section{Expériences professionnelles}


\cventry{Sep 2021 - Présent}{Animateur d'ateliers numériques}{Tech Kids Academy}{Paris}{}{Enseignement des bases de la programmation à des groupes de jeunes (7-17 ans). Scratch, python, arduino, etc.}

\cventry{Jul 2019 - Jan 2020}{Animateur d'ateliers numériques}{Villette Makerz}{Paris}{}{Initiation à la programmation, animation d'ateliers publics sur le thème des nouvelles technologies (impression 3D, robotique...)}

\cventry{Mar 2020 - Jun 2020}{Stagiaire en biochimie}{Muséum National d'Histoire Naturelle}{Paris}{}{Stage bibliographique (pandémie oblige) sur l'étude métagénomique des peptides antimicrobiens et leurs effets sur le microbiote intestinal.}

\cventry{Jan 2019 - Jun 2019}{Stagiaire en génétique moléculaire}{Institut Pasteur}{Paris}{}{Extraction d'ADN de S. pombe et détection de la méthylation en sortie de G0. Techniques utilisée: extraction d'ADN au phénol, FACS, Migration d'ADN sur gradient de sucrose}


%\cventry{years}{job title}{employer}{localization}{grade}{description}
%\cvline{name of item}{description}
%\cvcomputer{category 1}{XXX, YYY, ZZZ}{category 2}{XXX, YYY, ZZZ}

\section{Compétences Informatique et Langues}
\subsection{Informatique}
\cvcomputer{Langages}{\textbf{C++}:Bases solides\newline{}\textbf{Python}:Avancé\newline{}\textbf{C}:Bonnes bases\newline{}}{Logiciels}{\textbf{Git}:Bases\newline{}\textbf{Unity}:Intermédiaire\newline{}\textbf{Blender}:Intermédiaire\newline{}}{}
\subsection{Langues}

\cvline{Français}{Natif}

\cvline{Anglais}{C1}

\cvline{Espagnol}{B2}

\section{Centres d'intérêt}

\cvline{Loisirs}{Guitare\newline{}Astrophotographie\newline{}Arduino\newline{}Minecraft\newline{}Modélisation 3D\newline{}}

\end{document}